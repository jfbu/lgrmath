% -*- coding: utf-8; time-stamp-format: "%02d-%02m-%:y at %02H:%02M:%02S %Z" -*-
% Execute etex on this file to extract package files.
% See in README for more build instructions.
%<*dtx>
\def\dtxtimestamp {Time-stamp: <16-11-2022 at 15:12:59 CET>}
%%
%% Package: lgrmath
%% Greek in math mode via LGR font of one's choice (JFB)
%% Version: 1.0
%% License: LPPL 1.3c
%% Copyright (C) 2022 Jean-François B.
%% Repository: https://github.com/jfbu/lgrmath
%%
%</dtx>
%<*tex>
\def\lgrmathversion{1.0}
\def\lgrmathdate{2022/11/15}
%</tex>
%<*dtx>
\iffalse
%</dtx>
%<*tex>
%% This is a generated file. Run latexmk on this file lgrmath.tex then
%% run dvipdfmx on lgrmath.dvi to produce the documentation lgrmath.pdf,
%% with the package source code included.
%%
%% Customize as desired the class options and the two toggles below.
%%
%% See lgrmath.dtx for the copyright and the conditions for distribution
%% and/or modification of this Work.
%%
\NeedsTeXFormat{LaTeX2e}
\ProvidesFile{lgrmath.tex}%
[\lgrmathdate\space v\lgrmathversion\space 
       driver file for lgrmath documentation (JFB)]%
\PassOptionsToClass{a4paper,fontsize=11pt}{scrartcl}
\chardef\Withdvipdfmx \ifdefined\pdfoutput\ifnum\pdfoutput>0 0\else1\fi\else0\fi\relax
\chardef\NoSourceCode 0 % replace 0<space> by 1<space> for no source code
\input lgrmath.dtx
%%% Local Variables:
%%% mode: latex
%%% End:
%</tex>
%<*dtx>
\fi
\chardef\noetex 0
\ifx\numexpr\undefined\chardef\noetex 1 \fi
\ifnum\noetex=1 \chardef\extractfiles 0 % extract files, then stop
\else
    \ifx\ProvidesFile\undefined
      \chardef\extractfiles 0 % etex etc.. on lgrmath.dtx
    \else % latex/pdflatex on lgrmath.tex or on lgrmath.dtx
      \ifx\Withdvipdfmx\undefined
        % (lua|xe|pdf)latex run is on lgrmath.dtx, we will extract all files
        \chardef\extractfiles 1 % 1 = extract all and typeset doc
        \chardef\Withdvipdfmx
                \ifdefined\pdfoutput\ifnum\pdfoutput>0 0\else1\fi\else0\fi\relax
        \chardef\NoSourceCode 1 % 
        \NeedsTeXFormat{LaTeX2e}%
        \PassOptionsToClass{a4paper,fontsize=11pt}{scrartcl}%
      \else % latex run is on lgrmath.tex, 
        \chardef\extractfiles 2 % no extractions
      \fi
    \ProvidesFile{lgrmath.dtx}%
    [lgrmath source (\lgrmathversion\space of \lgrmathdate) (JFB)]%
    \fi
\fi
\ifnum\extractfiles<2 % extract files
\def\MessageDeFin{\newlinechar10 \let\Msg\message
\Msg{^^J}%
\Msg{********************************************************************^^J}%
\Msg{*^^J}%
\Msg{* To finish the installation you have to move the following^^J}%
\Msg{* file into a directory searched by TeX:^^J}%
\Msg{*^^J}%
\Msg{*\space\space\space\space lgrmath.sty^^J}%
\Msg{*^^J}%
\Msg{* To produce the documentation with source code included run latexmk^^J}%
\Msg{* on extracted lgrmath.tex and then dvipdfmx on lgrmath.dvi^^J}%
\Msg{*^^J}%
\Msg{* Happy TeXing!^^J}%
\Msg{*^^J}%
\Msg{********************************************************************^^J}%
}%
\begingroup
\input docstrip.tex
\askforoverwritefalse
\generate{\nopreamble
  \usepostamble\defaultpostamble
    \file{lgrmath.tex}{\from{lgrmath.dtx}{tex}}%
  \usepreamble\defaultpreamble
    \file{lgrmath.sty}{\from{lgrmath.dtx}{sty}}%
}
\endgroup
\fi % end of file extractions
\ifnum\extractfiles=0 
% tex/etex/xetex/etc on lgrmath.dtx, files are now extracted, stop
  \MessageDeFin\expandafter\end
\fi
% no file extractions if latex compilation was on lgrmath.tex
\ifdefined\MessageDeFin\AtEndDocument{\MessageDeFin}\fi
\ifnum\Withdvipdfmx>0
\documentclass [dvipdfmx]{scrartcl}
\else
\documentclass {scrartcl}
\fi
\tracinglostchars=2

% use doc with v2 because hyperref option clash with hypdoc
% no time to read documentation now
\usepackage{doc}[=v2]
% but the v2 does not define \cs, it got defined in ltxdoc
\def\cmd#1{\cs{\expandafter\cmd@to@cs\string#1}}
\def\cmd@to@cs#1#2{\char\number`#2\relax}
\DeclareRobustCommand\cs[1]{\texttt{\bslash#1}}
\AtBeginDocument{%
\pdfstringdefDisableCommands{\def\cs#1{\textbackslash\detokenize{#1}}}%
}%
% these are not defined by doc package
\providecommand\marg[1]{%
  {\ttfamily\char`\{}\meta{#1}{\ttfamily\char`\}}}
\providecommand\oarg[1]{%
  {\ttfamily[}\meta{#1}{\ttfamily]}}
\providecommand\parg[1]{%
  {\ttfamily(}\meta{#1}{\ttfamily)}}

\CodelineNumbered
\AtBeginDocument{\MakeShortVerb\|}
\ifnum\NoSourceCode=1
  \OnlyDescription
\fi
\usepackage[LGR,T1]{fontenc}

\usepackage{hologo}
%% \hologoFontSetup{general=\upshape\rmfamily}
\DeclareRobustCommand*{\pdfLaTeX}{\hologo{pdfLaTeX}}%
% \DeclareRobustCommand*{\eTeX}{\hologo{eTeX}}%
% \DeclareRobustCommand*{\LuaTeX}{\hologo{LuaTeX}}%
% \DeclareRobustCommand*{\LuaLaTeX}{\hologo{LuaLaTeX}}%
% \DeclareRobustCommand*{\XeTeX}{\hologo{XeTeX}}%
% \DeclareRobustCommand*{\XeLaTeX}{\hologo{XeLaTeX}}%

\usepackage{geometry}
\usepackage[english]{babel}

\usepackage{color}
\definecolor{joli}{RGB}{225,95,0}
\definecolor{JOLI}{RGB}{225,95,0}
\definecolor{urlcolor}{RGB}{38,128,192}

\usepackage{colorframed}
\definecolor{shadecolor}{RGB}{223,219,195}

\usepackage{hyperref}
\hypersetup{%
linktoc=all,%
breaklinks=true,%
colorlinks=true,%
linkcolor={red},%
urlcolor={urlcolor},
pdfauthor={Jean-François B.},%
pdftitle={The lgrmath package},%
pdfstartview=FitH,%
pdfpagemode=UseNone,%
}
\usepackage{hypcap}
\usepackage{bookmark}
\usepackage{enumerate}

\usepackage{mlmodern}
\DeclareEncodingSubset{TS1}{mlmtt}{0}% fix \textasciigrave related latex2e#905

% useless with doc.sty
% \usepackage{upquote}
% as \verb, verbatim and macrocode do not use \@noligs
\begingroup
\catcode`\'\active\catcode`\`\active
\gdef\MakePrivateLetters{\makeatletter\def`{\textasciigrave}\def'{\textquotesingle}}
\endgroup
% https://github.com/latex3/latex2e/issues/953, for the record

\usepackage{xspace}

\newcommand\ctanpkg[1]{\href{https://ctan.org/pkg/#1}{#1}}

% Vendredi 11 novembre 2022 : impossible d'utiliser \hypersetup{urlcolor=joli}
% dans \section (même sans le \texorpdfstring).
% même si la macro \lgrmath est déclarée robuste mais avec \protected c'est bon
% \DeclareRobustCommand\lgrmath{{\hypersetup{urlcolor=joli}lgrmath}}% <-- BAD
% ceci ok (mais il faudra ajouter \texorpdfstring)
% \protected\def\lgrmath{{\hypersetup{urlcolor=joli}lgrmath}}%        <-- OK
% L'erreur est
%     Argument of \KVS@@CommaComma has an extra }
% mais je n'ai pas cherché à en savoir plus.  De plus dans le code ci-dessous
% on avait plutôt un stack overflow dû à boucle infinie.

% le \texorpdfstring ne sert à rien si \protected dans \section
\protected\def\lgrmath{%
   {\hypersetup{urlcolor=joli}\bfseries \ctanpkg{lgrmath}}\xspace
}
\pdfstringdefDisableCommands{\def\lgrmath{lgrmath\xspace}}

\usepackage{centeredline}
\usepackage{lgrmath}

\begin{document}\renewcommand\familydefault\sfdefault
\begin{center}
 {\Large The \lgrmath package}\\
 \textsc{Jean-François B.}\\
 \texttt{jfbu (at) free (dot) fr}\\
\footnotesize
  This is \lgrmathversion\ (\lgrmathdate) from
  source file having \dtxtimestamp\\
  Report issues at \url{https://github.com/jfbu/lgrmath/issues}
\end{center}

{
\addtocontents{toc}{\protect\hypersetup{hidelinks}}
\addtocontents{lot}{\protect\hypersetup{hidelinks}}

\tableofcontents

\listoftables
}

\section{Description}

The \lgrmath package sets the Greek letters {in math mode} (\textbf{only}) to use glyphs
from the LGR-encoded font of one's choice.

Thus \lgrmath is for people who want \emph{only} to adjust Greek letters in
math mode (and easily configure usage of upright or italic/slanted shapes),
perhaps in the context of having changed Latin letters as well, e.g.\@ from
using the \ctanpkg{frenchmath}%
%
\footnote{Antoine \textsc{Missier}, \emph{Typesetting mathematics according to French rules}, \url{https://ctan.org/pkg/frenchmath}.}
%
package which makes uppercase Latin letters in
math mode render upright, among quite a few other adjustments tailored for
French mathematical typesetting, or the \ctanpkg{mathastext}%
%
\footnote{Verfasser, \emph{Use the text font in math mode}, \url{https://ctan.org/pkg/mathastext}.}
%
package.
Actually \lgrmath is in part inspired from this latter package |LGRgreek|
option and \cs{MTgreekfont} command.  But \lgrmath currently does not
incorporate a mechanism for defining and using multiple math versions, each one
with its own font for Greek letters, as is already provided by
\ctanpkg{mathastext}.

The package is also related to \ctanpkg{libgreek}%
%
\footnote{Verfasser, \emph{Greek letters in math mode from Libertinus or Linux
    Libertine/Biolinum}, \url{https://ctan.org/pkg/libgreek}.},
%
also by the author, and shares most of its codebase, after dropping matters
related to |libgreek-legacy|, and the |scale| option which can not be
implemented generically.

The Greek letters all come with \cs{...up} and \cs{...it} named variants, and
whether ``bare'' control sequences map to the `|up|' or `|it|' ones can be
configured via package options, even midway in the document via
\cs{lgrmathsetup}.  Further, the package optionally defines two math
alphabets \cs{lgrmathup} and \cs{lgrmathit}.  What `|up|' and `|it|'
actually mean can be configured using the |upshape| and |itshape| keys at
package loading time.


\section{Options of the \lgrmath package}

\begin{table}[htbp]
\capstart
\DeleteShortVerb\|% a effet global !
\let\s\string
\ttfamily
\lgrmathsetup{style=fReNcH}% testing lowercasing works

\centering

\begin{tabular}{|lc|lc|lc|lc|}
\hline
\s\Alpha&$\Alpha$    &\s\Nu&$\Nu$          &\s\alpha&$\alpha$    &\s\nu&$\nu$\\
\s\Beta&$\Beta$      &\s\Xi&$\Xi$          &\s\beta&$\beta$      &\s\xi&$\xi$\\
\s\Gamma&$\Gamma$    &\s\Omicron&$\Omicron$&\s\gamma&$\gamma$    &\s\omicron&$\omicron$\\
\s\Delta&$\Delta$    &\s\Pi&$\Pi$          &\s\delta&$\delta$    &\s\pi&$\pi$\\
\s\Epsilon&$\Epsilon$&\s\Rho&$\Rho$        &\s\epsilon&$\epsilon$&\s\rho&$\rho$\\
\s\Zeta&$\Zeta$      &\s\Sigma&$\Sigma$    &\s\zeta&$\zeta$      &\s\sigma&$\sigma$\\
\s\Eta&$\Eta$        &\s\Tau&$\Tau$        &\s\eta&$\eta$        &\s\tau&$\tau$\\
\s\Theta&$\Theta$    &\s\Upsilon&$\Upsilon$&\s\theta&$\theta$    &\s\upsilon&$\upsilon$\\
\s\Iota&$\Iota$      &\s\Phi&$\Phi$        &\s\iota&$\iota$      &\s\phi&$\phi$\\
\s\Kappa&$\Kappa$    &\s\Chi&$\Chi$        &\s\kappa&$\kappa$    &\s\chi&$\chi$\\
\s\Lambda&$\Lambda$  &\s\Psi&$\Psi$        &\s\lambda&$\lambda$  &\s\psi&$\psi$\\
\s\Mu&$\Mu$          &\s\Omega&$\Omega$    &\s\mu&$\mu$          &\s\omega&$\omega$\\
\hline
\end{tabular}

\medskip

\begin{tabular}{lclclc}
\s\varsigma&$\varsigma$       & \s\digamma&$\digamma$ & \s\varSigma&$\varSigma$ \\
\s\varvarsigma&$\varvarsigma$ & \s\koppa&$\koppa$     & \s\Sampi&$\Sampi$       \\
\s\sampi&$\sampi$             &         &             & \s\Digamma&$\Digamma$
\end{tabular}
\caption{Greek letters, upright shapes, default family}
\label{table:upright}
\MakeShortVerb\|
\end{table}

\begin{table}[htbp]
\capstart
\DeleteShortVerb\|% a effet global !
\let\s\string
\ttfamily
\lgrmathsetup{style=ISO}

\centering
\begin{tabular}{|lc|lc|lc|lc|}
\hline
\s\Alpha&$\Alpha$    &\s\Nu&$\Nu$          &\s\alpha&$\alpha$    &\s\nu&$\nu$\\
\s\Beta&$\Beta$      &\s\Xi&$\Xi$          &\s\beta&$\beta$      &\s\xi&$\xi$\\
\s\Gamma&$\Gamma$    &\s\Omicron&$\Omicron$&\s\gamma&$\gamma$    &\s\omicron&$\omicron$\\
\s\Delta&$\Delta$    &\s\Pi&$\Pi$          &\s\delta&$\delta$    &\s\pi&$\pi$\\
\s\Epsilon&$\Epsilon$&\s\Rho&$\Rho$        &\s\epsilon&$\epsilon$&\s\rho&$\rho$\\
\s\Zeta&$\Zeta$      &\s\Sigma&$\Sigma$    &\s\zeta&$\zeta$      &\s\sigma&$\sigma$\\
\s\Eta&$\Eta$        &\s\Tau&$\Tau$        &\s\eta&$\eta$        &\s\tau&$\tau$\\
\s\Theta&$\Theta$    &\s\Upsilon&$\Upsilon$&\s\theta&$\theta$    &\s\upsilon&$\upsilon$\\
\s\Iota&$\Iota$      &\s\Phi&$\Phi$        &\s\iota&$\iota$      &\s\phi&$\phi$\\
\s\Kappa&$\Kappa$    &\s\Chi&$\Chi$        &\s\kappa&$\kappa$    &\s\chi&$\chi$\\
\s\Lambda&$\Lambda$  &\s\Psi&$\Psi$        &\s\lambda&$\lambda$  &\s\psi&$\psi$\\
\s\Mu&$\Mu$          &\s\Omega&$\Omega$    &\s\mu&$\mu$          &\s\omega&$\omega$\\
\hline
\end{tabular}
\medskip

\begin{tabular}{lclclc}
\s\varsigma&$\varsigma$       & \s\digamma&$\digamma$ & \s\varSigma&$\varSigma$ \\
\s\varvarsigma&$\varvarsigma$ & \s\koppa&$\koppa$     & \s\Sampi&$\Sampi$       \\
\s\sampi&$\sampi$             &         &             & \s\Digamma&$\Digamma$
\end{tabular}
\caption{Greek letters, italic shapes, default family}
\label{table:italic}
\MakeShortVerb\|
\end{table}

Here are the options recognized by the package:
\begin{description}
\item[font=\meta{font\textunderscore name}] This specifies which font (font family in the
  sens of the \LaTeX{} font selection scheme) to use.  It defaults to |lmr|.

  In \autoref{table:upright} and \autoref{table:italic} we display the
  glyphs from this default font |lmr| in LGR encoding, available to \LaTeX{}
  thanks to the
  support files from the package (in the sense of CTAN or \TeX Live, not of a
  \LaTeX{} document) \ctanpkg{cbfonts-fd}.%
%
  \footnote{Claudio~\textsc{Beccari}, \emph{\LaTeX{} font description files for the CB Greek fonts}, \url{https://ctan.org/pkg/cbfonts-fd}.}
%
  It is recommended to user to have a look at its
  documentation\centeredline{|texdoc cbfonts|}
  in particular the section on Customizations which mentions alternate shapes
  (such as |rs|, |ro|, |li|, |iv|, |uv| --- those last two are actually for
  sans-serif |lmss| ---, and there are also comments relative to the series)
  and use appropriately the |upshape|, |itshape|, |series| and |boldseries|
  \lgrmath keys which are documented next.

  The allowable names \meta{font\textunderscore name}'s are those |foo| for
  which a file |LGRfoo.fd| or |lgrfoo.fd| exists on the system.

  The above remarks about customization apply generally to all fonts, try to
  see if there is some documentation associated with the font you choose.
  Ultimate experts will look into the |.fd| files to see (for example) if there
  is some interface to rescale the fonts by some factor.

  Here is now a list of suitable such font definition files from which you can
  extract usable font family names.  This has been obtained via exercising the
  Unix |find| utility in a \TeX Live 2022 installation (possibly only partial).
  To test a font the package provides \cs{lgrmathgreektable} and
  \cs{lgrmathgreektableextra} which are documented in the next section.

  \begin{snugshade}
\catcode`\'\active\catcode`\`\active
\def\MacroFont{\ttfamily\footnotesize\def'{\textquotesingle}\def`{\textasciigrave}}
\catcode\string`\`12 \catcode`\' 12
\begin{verbatim}
in directory  /usr/local/texlive/2022/texmf-dist/tex/latex we execute
    find . -name 'LGR*fd'
and then rearrange somewhat the output to put it in alphabetical order,
and gain some space horizontally so as to obtain a two-column display
Naturally in many instances the variaous -TLF, -OsF, and so on, refer
to options of digit characters and have no impact on the Greek letters,
nevertheless I kept all filenames, just pick one, drop LGR and .fd parts.

./alegreya/                                     ./librefranklin/
    LGRAlegreya-Inf.fd                              LGRLibreFranklin-Sup.fd
    LGRAlegreya-LF.fd                               LGRLibreFranklin-TLF.fd
    LGRAlegreya-OsF.fd                          ./linguisticspro/
    LGRAlegreya-Sup.fd                              LGRLinguisticsPro-LF.fd
    LGRAlegreya-TLF.fd                              LGRLinguisticsPro-OsF.fd
    LGRAlegreya-TOsF.fd                         ./nimbus15/
    LGRAlegreyaSans-Inf.fd                          LGRNimbuSans.fd
    LGRAlegreyaSans-LF.fd                           LGRNimbusMono.fd
    LGRAlegreyaSans-OsF.fd                          LGRNimbusMonoN.fd
    LGRAlegreyaSans-Sup.fd                          LGRNimbusSerif.fd
    LGRAlegreyaSans-TLF.fd                      ./noto/
    LGRAlegreyaSans-TOsF.fd                         LGRNotoSans-LF.fd
./clara/                                            LGRNotoSans-OsF.fd
    LGRClara-Sup.fd                                 LGRNotoSans-Sup.fd
    LGRClara-TLF.fd                                 LGRNotoSans-TLF.fd
    LGRClara-TOsF.fd                                LGRNotoSans-TOsF.fd
./cochineal/                                        LGRNotoSansMono-Sup.fd
    LGRCochineal-LF.fd                              LGRNotoSansMono-TLF.fd
    LGRCochineal-OsF.fd                             LGRNotoSansMono-TOsF.fd
    LGRCochineal-TLF.fd                             LGRNotoSerif-LF.fd
    LGRCochineal-TOsF.fd                            LGRNotoSerif-OsF.fd
./comfortaa/                                        LGRNotoSerif-Sup.fd
    LGRcomfortaa.fd                                 LGRNotoSerif-TLF.fd
./dejavu/                                           LGRNotoSerif-TOsF.fd
    LGRDejaVuSans-TLF.fd                        ./oldstandard/
    LGRDejaVuSansCondensed-TLF.fd                   LGROldStandard-Sup.fd
    LGRDejaVuSansMono-TLF.fd                        LGROldStandard-TLF.fd
    LGRDejaVuSerif-TLF.fd                       ./opensans/
    LGRDejaVuSerifCondensed-TLF.fd                  LGRopensans-LF.fd
./domitian/                                         LGRopensans-OsF.fd
    LGRDomitian-Inf.fd                              LGRopensans-TLF.fd
    LGRDomitian-Sup.fd                              LGRopensans-TOsF.fd
    LGRDomitian-TLF.fd                          ./plex/
    LGRDomitian-TOsF.fd                             LGRIBMPlexSans-Sup.fd
./droid/                                            LGRIBMPlexSans-TLF.fd
    LGRdroidsans.fd                             ./roboto/
    LGRdroidsansmono.fd                             LGRRoboto-LF.fd
    LGRdroidserif.fd                                LGRRoboto-OsF.fd
./ebgaramond/                                       LGRRoboto-TLF.fd
    LGREBGaramond-Inf.fd                            LGRRoboto-TOsF.fd
    LGREBGaramond-LF.fd                             LGRRobotoMono-TLF.fd
    LGREBGaramond-OsF.fd                            LGRRobotoSerif-LF.fd
    LGREBGaramond-Sup.fd                            LGRRobotoSerif-OsF.fd
    LGREBGaramond-TLF.fd                            LGRRobotoSerif-Sup.fd
    LGREBGaramond-TOsF.fd                           LGRRobotoSerif-TLF.fd
    LGREBGaramondInitials-TLF.fd                    LGRRobotoSerif-TOsF.fd
./fira/                                             LGRRobotoSlab-TLF.fd
    LGRFiraMono-Sup.fd                          ./sourcesanspro/
    LGRFiraMono-TLF.fd                              LGRSourceSansPro-Dnom.fd
    LGRFiraMono-TOsF.fd                             LGRSourceSansPro-Inf.fd
    LGRFiraSans-LF.fd                               LGRSourceSansPro-LF.fd
    LGRFiraSans-OsF.fd                              LGRSourceSansPro-Numr.fd
    LGRFiraSans-Sup.fd                              LGRSourceSansPro-OsF.fd
    LGRFiraSans-TLF.fd                              LGRSourceSansPro-Sup.fd
    LGRFiraSans-TOsF.fd                             LGRSourceSansPro-TLF.fd
./garamond-libre/                                   LGRSourceSansPro-TOsF.fd
    LGRGaramondLibre-Inf.fd                     ./step/
    LGRGaramondLibre-LF.fd                          LGRSTEP-Inf.fd
    LGRGaramondLibre-OsF.fd                         LGRSTEP-Sup.fd
    LGRGaramondLibre-Sup.fd                         LGRSTEP-TLF.fd
./gofonts/                                          LGRSTEP-TOsF.fd
    LGRGo-TLF.fd                                ./stepgreek/
    LGRGoMono-TLF.fd                                LGRSTEPGreekTest-Sup.fd
./lato/                                             LGRSTEPGreekTest-TLF.fd
    LGRlato-LF.fd                                   LGRSTEPGreekTest-TOsF.fd
    LGRlato-OsF.fd                              ./theanodidot/
    LGRlato-TLF.fd                                  LGRTheanoDidot-TLF.fd
    LGRlato-TOsF.fd                                 LGRTheanoDidot-TOsF.fd
./libertinegc/                                  ./theanomodern/
    LGRLinuxLibertineT-LF.fd                        LGRTheanoModern-TLF.fd
    LGRLinuxLibertineT-OsF.fd                       LGRTheanoModern-TOsF.fd
    LGRLinuxLibertineT-TLF.fd                   ./theanooldstyle/
    LGRLinuxLibertineT-TOsF.fd                      LGRTheanoOldStyle-TLF.fd
./libertinus-type1/                                 LGRTheanoOldStyle-TOsF.fd
    LGRLibertinusSans-LF.fd        LGRLibertinusSerif-TLF.fd
    LGRLibertinusSans-OsF.fd       LGRLibertinusSerif-TOsF.fd
    LGRLibertinusSans-Sup.fd       LGRLibertinusSerifDisplay-LF.fd
    LGRLibertinusSans-TLF.fd       LGRLibertinusSerifDisplay-OsF.fd
    LGRLibertinusSans-TOsF.fd      LGRLibertinusSerifDisplay-Sup.fd
    LGRLibertinusSerif-LF.fd       LGRLibertinusSerifDisplay-TLF.fd
    LGRLibertinusSerif-OsF.fd      LGRLibertinusSerifDisplay-TOsF.fd
    LGRLibertinusSerif-Sup.fd      LGRLibertinusSerifInitials-TLF.fd

And now for more, with lowercase `lgr' filenames: find . -name 'lgr*fd'

./txfontsb/lgrtxr.fd                     ./cm-lgc/lgrfcm.fd
./txfontsb/lgrtxrc.fd                    ./cm-lgc/lgrfct.fd
./txfontsb/lgrtxry.fd                    ./cm-lgc/lgrfcs.fd
./txfontsb/lgrtxryc.fd                   ./epigrafica/lgrepigrafica.fd
./gfsbodoni/lgrbodoni.fd                 ./gfssolomos/lgrsolomos.fd
./lxfonts/lgrllcmtt.fd                   ./tempora/lgrtempora-tlf.fd
./lxfonts/lgrllcmss.fd                   ./tempora/lgrtempora-tosf.fd
./kerkis/lgrkfn.fd                       ./gfscomplutum/lgrcomplutum.fd
./kerkis/lgrmaksf.fd                     ./gfsartemisia/lgrartemisiaeuler.fd
./kerkis/lgrmak.fd                       ./gfsartemisia/lgrartemisia.fd
./cbfonts-fd/lgrcmro.fd                  ./gentium-tug/lgrgentiumbook.fd
./cbfonts-fd/lgrcmss.fd                  ./gentium-tug/lgrgentium.fd
./cbfonts-fd/lgrlmr.fd                   ./gfsbaskerville/lgrgfsbaskerville.fd
./cbfonts-fd/lgrlcmtt.fd                 ./miama/lgrfmm.fd
./cbfonts-fd/lgrlmtt.fd                  ./gfsneohellenic/lgrneohellenic.fd
./cbfonts-fd/lgrlmss.fd                  ./gfsdidot/lgrudidot.fd
./cbfonts-fd/lgrlmro.fd                  ./gfsporson/lgrporson.fd
./cbfonts-fd/lgrlcmss.fd
./cbfonts-fd/lgrcmtt.fd
./cbfonts-fd/lgrcmr.fd
\end{verbatim}
\end{snugshade}
\item[upshape=\meta{shape}] Declares the shape to be used by the \cs{...up}
  Greek letters and the \cs{lgrmathup} math alphabet.  Defaults to `|n|'
  (without the quotes).
\item[itshape=\meta{shape}] Declares the shape to be used by the \cs{...it}
  Greek letters and the \cs{lgrmathit} math alphabet.  Defaults to `|it|'.
\DeleteShortVerb{\|}
\item[style=\meta{\upshape\ttfamily ISO|UP|TeX}] specifies the shape style
  of the Greek letters. \MakeShortVerb{\|}

  |ISO| means `italic' for lowercase
  and uppercase, |UP| means `upright' for lowercase and uppercase, |TeX|
  means `italic' for lowercase and `upright' for uppercase.  The lowercase
  forms |iso|, |up| and |tex| are also accepted (or any mixed case).

  One can also use |French| or |french| as an alias to |UP| or |up|.

  This option will
  override any |greek| or |Greek| option. The package defaults to |style=TeX|.

  What `upright' and `italic' mean is configured by the |upshape| and
  |itshape| respective settings.\DeleteShortVerb\|
\item[greek=\meta{\ttfamily\upshape up|it|...}] Says\MakeShortVerb{\|} whether
  Greek letters will be `upright' or `italic' i.e.\@ whether they obey the
  |upshape| or |itshape| setting, i.e.\@ whether \cs{alpha} et al.\@ are
  \cs{let} to \cs{alphaup} (et al.) or to \cs{alphait} (et
  al.).\MakeShortVerb{\|}

  So |greek=it| is like |style=ISO|, and |greek=up| is like |style=French|.

  Other
  shape values, such as `|n|' and `|sl|' or even `|sc|', are accepted.  For
  more details, see the explanations for |Greek|.  For example |greek=n| is
  like |style=UP|.

  This option is ignored if |style| is used (order does not matter).
  \DeleteShortVerb\|
\item[Greek=\meta{\ttfamily\upshape up|it|...}] Says\MakeShortVerb{\|}
  whether uppercase Greek
  letters (and only them) will be `upright' or `italic' i.e.\@ whether they
  use |upshape| or |itshape|, i.e.\@ whether \cs{Alpha} et al.\@ are \cs{let}
  to \cs{Alphaup} (et al.) or to \cs{Alphait} (et al.).\MakeShortVerb{\|}

  So to obtain lowercase to be `upright' and uppercase to be `italic', use
  |greek=up| and then |Greek=it| (|Greek| must appear after |greek| else it will be
  shadowed by it).

  This option, like the  |greek| option, is ignored if the |style| option is used.

  Other
  shape values, such as `|n|' and `|sl|', are accepted.  They will then
  override the |upshape| setting for it to match it.  For example |Greek=sc|
  will force |upshape| to be |sc|, because the assumed style is the \TeX{} one
  of italic lowercase and upright uppercase, so setting the shape of uppercase
  must update the |upshape| value.

\item[series=\meta{series}] This tells which series to use. The default is the
  value of \cs{seriesdefault} at the time of loading the package.  There is no
  interface to configure distinct series for the `upright' and `italic'
  shapes.

\item[boldseries=\meta{series}] This tells which series to use in bold
  math. Default is \cs{bfdefault} at the time of loading the package.  There
  is no interface to configure distinct series for the `upright' and `italic'
  shapes.
\item[alphabets] Says whether to define \cs{lgrmathup} and \cs{lgrmathit}.
\end{description}

\section{Commands of the \lgrmath package}

Here are the commands defined by the package:
\begin{description}
\item[\cs{lgrmathsetup}\marg{key=value,...}] The only allowed keys are
  |style|, |greek| and |Greek|.  And for the latter two only the values |up|
  or |it| should be used (or values matching the |upshape| or |itshape|
  settings), as it is only possible after package loading time to toggle
  between `upright' and `italic' depending on whether the letter is uppercase
  or lowercase, but one can not switch to an altogether different shape as
  this would require re-declaring the symbol fonts.

  If the |style| key is used, then |greek/Greek| are ignored.  However,
  one can always naturally reuse later \cs{lgrmathsetup} using only the
  |greek| and/or |Greek| keys.
\item[\cs{lgrmathup}] This is a math alphabet.  It is defined only if the
  package received the |alphabets| option.
\item[\cs{lgrmathit}] This is a math alphabet.  It is defined only if the
  package received the |alphabets| option.

\item[\cs{lgrmathgreektable}\marg{family}\marg{series}\marg{shape}] Produces a
  tabular display of the Greek letters available with this font.  Here is for
  example using%
%
  \centeredline{|\lgrmathgreektable{Alegreya-TLF}{regular}{n}|}
%
  \centeredline{\lgrmathgreektable{Alegreya-TLF}{regular}{n}}

  We used |regular|
  for the \marg{series} mandatory argument after seeing Font Info messages in
  the |.log| file about the |m| series not being available and being
  substituted for by |regular|, so we used |regular| to avoid those messages.
\item[\cs{lgrmathgreektableextra}\marg{family}\marg{series}\marg{shape}] 
  Produces a tabular with eight additional glyphs.  Here is an example, using
%
  \centeredline{|\lgrmathgreektableextra{LibertinusSans-TLF}{m}{n}|}
%
  \centeredline{\lgrmathgreektableextra{LibertinusSans-TLF}{m}{n}} Beware that if
  we had forgotten the |-TLF| suffix, the font would have been substituted in
  favour of fall-back |lmr| by \LaTeX.  Always check log for font substitutions
  messages...

  And see also third remark next.
\end{description}

Miscellaneous remarks:
\begin{enumerate}
\item Even if not receiving the option |alphabets|, the package will declare
  all Greek letters to be of type \cs{mathalpha}.
\item The \lgrmath package ignores global class options.  It handles only
  options originating from the \cs{usepackage} preamble declaration (or some
  options handed over via \cs{PassOptionsToPackage} or options passed to
  \cs{lgrmathsetup} in the preamble or body).
\item The ``extra glyphs'' will not be available will all fonts.  Here is for example
  with |Alegreya-TLF|:
\centeredline{\lgrmathgreektableextra{Alegreya-TLF}{regular}{n}}
  Adding |\tracinglostchars=3| will cause \TeX\ to raise an error in case such
  missing characters are encountered.
\end{enumerate}

This is the end of the user documentation.  The next section is a code listing
with some comments for the advanced users.

\StopEventually{\end{document}\endinput}

\cleardoublepage
%\newgeometry{hmarginratio=3:2}
\small

\makeatletter

\section{Implementation of the \lgrmath package}

% https://github.com/latex3/latex2e/issues/563
\AddToHook{env/macrocode/after}{\@nobreakfalse}
\AddToHook{env/macrocode/begin}{\partopsep0pt\relax}

% The catcode hackery next is to avoid to have the guard be listed
% in the commented source code... (here <*legacy>)
% (c) 2012/11/19 Jean-François B. ;-)

\def\gardesactifs {\catcode`\<=0 \catcode`\>=11 \catcode`\*=11 \catcode`\/=11 }
\def\gardesinactifs {\catcode`\<=12 \catcode`\>=12 \catcode`\*=12 \catcode`\/=12 }
\gardesactifs
\let</dtx>\relax
\let<*sty>\gardesinactifs

\MakePercentIgnore
%</dtx>
%<*sty>
%    \begin{macrocode}
\NeedsTeXFormat{LaTeX2e}
\ProvidesPackage{lgrmath}
         [2022/11/15 1.0 Greek in math mode via LGR font of one's choice (JFB)]
%    \end{macrocode}
% We will use \ctanpkg{kvoptions} to handle options with |key=value| syntax.
%    \begin{macrocode}
\RequirePackage{kvoptions}
%    \end{macrocode}
% To minimize the author's task, we keep close to libgreek.sty code with minimal
% adaptations.  In particular I decided to keep the fact that |style| option
% makes the |Greek| and |greek| options ignored.  But there are some
% complications originating in the addition of the \cs{lgrmathsetup}, which
% requires to keep a trace of various things, for example if |style| option is
% used at package level and then later on using \cs{lgrmathsetup} the user
% employs the |Greek/greek| options.
%
% This package assigns two symbol fonts, one for upright, the other one for
% italic-like.
%
% The |upshape| and |itshape| keys allow to configure what the \cs{...up}
% and \cs{...it} macros will actually use as shapes.
%    \begin{macrocode}
\def\lgrmath@fontfamily{lmr}
\def\lgrmath@scale{1}
\def\lgrmath@upshape{n}
\def\lgrmath@itshape{it}
\newif\iflgrmath@upper@up\lgrmath@upper@uptrue
\newif\iflgrmath@lower@up
\edef\lgrmath@series{\seriesdefault}
\edef\lgrmath@boldseries{\bfdefault}
\def\lgrmath@upper@shape{\lgrmath@upshape}
\def\lgrmath@lower@shape{\lgrmath@itshape}
\def\lgrmath@style{TeX}
\newif\iflgrmath@sty
%    \end{macrocode}
% We use the \ctanpkg{keyval} interface mostly to not have to rework
% everything, if at all possible, into the \ctanpkg{kvoptions} declarative
% interface.  It is a very good thing that the latter package can be used
% without forcing on the user its own declarative interface... however this
% means that inputs such as |serif=false| do not generate errors but are
% handled exactly the same as |serif=true|...
%    \begin{macrocode}
\define@key{lgrmath}{font}[lmr]{\def\lgrmath@fontfamily{#1}}
\define@key{lgrmath}{upshape}{\edef\lgrmath@upshape{#1}}
\define@key{lgrmath}{itshape}{\edef\lgrmath@itshape{#1}}
%    \end{macrocode}
% Compared to |libgreek| I decide to use \cs{lowercase} and allow more variety
% in the input.
%    \begin{macrocode}
\define@key{lgrmath}{style}{%
   \edef\lgrmath@style{#1}%
   \lowercase\expandafter{\expandafter\def\expandafter\lgrmath@style
               \expandafter{\lgrmath@style}}%
   \lgrmath@stytrue
}
\define@key{lgrmath}{Greek}{\edef\lgrmath@upper@shape{#1}}
%    \end{macrocode}
% Attention to not introduce a space token, as this may be used via
% \cs{lgrmathsetup} in document body.
%    \begin{macrocode}
\define@key{lgrmath}{greek}{\edef\lgrmath@lower@shape{#1}%
                             \edef\lgrmath@upper@shape{#1}}
\define@key{lgrmath}{series}{\edef\lgrmath@series{#1}}
\define@key{lgrmath}{boldseries}{\edef\lgrmath@boldseries{#1}}
%    \end{macrocode}
% The single truly Boolean option...
%    \begin{macrocode}
\DeclareBoolOption[false]{alphabets}
%    \end{macrocode}
% We need some auxiliaries to handle the |style| values.  As mentioned
% already, some extra stuff is executed for reasons of various scenarii
% with \cs{lgrmathsetup}.
%    \begin{macrocode}
\def\lgrmath@style@iso{%
    \lgrmath@upper@upfalse
    \lgrmath@lower@upfalse
    \let\lgrmath@upper@shape\lgrmath@itshape
    \let\lgrmath@lower@shape\lgrmath@itshape
}
\def\lgrmath@style@french{%
    \lgrmath@upper@uptrue
    \lgrmath@lower@uptrue
    \let\lgrmath@upper@shape\lgrmath@upshape
    \let\lgrmath@lower@shape\lgrmath@upshape
}
\let\lgrmath@style@up\lgrmath@style@french
\def\lgrmathk@style@tex{%
    \lgrmath@upper@uptrue
    \lgrmath@lower@upfalse
    \let\lgrmath@upper@shape\lgrmath@upshape
    \let\lgrmath@lower@shape\lgrmath@itshape
}
%    \end{macrocode}
% This always resets the \cs{iflgrmath@sty} to false for \cs{lgrmathsetup}
% being usable with |greek| and |Greek| keys.
%    \begin{macrocode}
\def\lgrmath@process@style{%
   \lgrmath@styfalse
   \ifcsname lgrmath@style@\lgrmath@style\endcsname
     \csname lgrmath@style@\lgrmath@style\endcsname
   \else
     \PackageWarning{lgrmath}{Unknown (here, lowercased) style `\lgrmath@style'}%
   \fi
}
%    \end{macrocode}
% This stuff is a bit involved.
%    \begin{macrocode}
\def\lgrmath@process@shapes{%
   \edef\lgrmath@upper@shape{\lgrmath@upper@shape}%
   \edef\lgrmath@lower@shape{\lgrmath@lower@shape}%
   \ifx\lgrmath@upper@shape\lgrmath@upshape
      \lgrmath@upper@uptrue
   \else
      \ifx\lgrmath@upper@shape\lgrmath@itshape
         \lgrmath@upper@upfalse
      \else
   \expandafter\in@\expandafter{\expandafter.\lgrmath@upper@shape,}{.up,}%
   \ifin@\lgrmath@upper@uptrue
   \else
     \expandafter\in@\expandafter{\expandafter.\lgrmath@upper@shape,}{.it,}%
     \ifin@\lgrmath@upper@upfalse
     \else
        \lgrmath@process@upper@lastresort
   \fi\fi\fi\fi
   \ifx\lgrmath@lower@shape\lgrmath@itshape
      \lgrmath@lower@upfalse
   \else
      \ifx\lgrmath@lower@shape\lgrmath@upshape
         \lgrmath@lower@uptrue
      \else
   \expandafter\in@\expandafter{\expandafter.\lgrmath@lower@shape,}{.it,}%
   \ifin@\lgrmath@lower@upfalse
   \else
     \expandafter\in@\expandafter{\expandafter.\lgrmath@lower@shape,}{.up,}%
     \ifin@\lgrmath@lower@uptrue
     \else
        \lgrmath@process@lower@lastresort
   \fi\fi\fi\fi
}% 
\def\lgrmath@process@upper@lastresort{%
    \lgrmath@upper@uptrue
    \let\lgrmath@upshape\lgrmath@upper@shape
}
\def\lgrmath@process@lower@lastresort{%
    \lgrmath@lower@upfalse
    \let\lgrmath@itshape\lgrmath@lower@shape
}
%    \end{macrocode}
% The fact that packages may be handed global options is rather dangerous.
% Fortunately \ctanpkg{kvoptions} has an interface to handle only local options.
%    \begin{macrocode}
\ProcessLocalKeyvalOptions*
%    \end{macrocode}
% We now do the post-processing regarding the shape configuration after
% option parsing.  Once this is done we will reconfigure slightly
% \cs{lgrmath@process@shapes} for usability in the document preamble or body,
% after the symbol fonts have been declared.  As is well-known the \LaTeX{}
% interface to math fonts is full of  ``only-preamble'' restrictions.
%    \begin{macrocode}
\iflgrmath@sty
  \lgrmath@process@style
\else
  \lgrmath@process@shapes
\fi
\def\lgrmath@process@upper@lastresort{%
    \PackageWarning{lgrmath}{%
                       Too late for the shape `\lgrmath@upper@shape'\MessageBreak
                       originating in Greek or greek option. Ignored.\MessageBreak
                       Use `up' or `it'}%
}
\def\lgrmath@process@lower@lastresort{%
    \PackageWarning{lgrmath}{%
                       Too late for the shape `\lgrmath@lower@shape'\MessageBreak
                       originating in greek option. Ignored.\MessageBreak
                       Use `up' or `it'}%
}
\def\lgrmathsetup#1{%
    \setkeys{lgrmath}{#1}%
    \iflgrmath@sty\lgrmath@process@style\else\lgrmath@process@shapes\fi
    \lgrmath@setgreekcs
}
%    \end{macrocode}
% Almost all options must be restricted to the package loading time only.
%    \begin{macrocode}
\DisableKeyvalOption{lgrmath}{font}
\DisableKeyvalOption{lgrmath}{upshape}
\DisableKeyvalOption{lgrmath}{itshape}
\DisableKeyvalOption{lgrmath}{series}
\DisableKeyvalOption{lgrmath}{boldseries}
\DisableKeyvalOption{lgrmath}{alphabets}
%    \end{macrocode}
% Declarations of the two symbol fonts, one for `upright' (or whatever is
% configured by the |upshape| key), one for `italic' (or whatever is configured
% by the |itshape| key).  One can not specify distinct series, both
% `upright' and `italic' use the same font series.  This could be added but I
% doubt anyone will use the package to start with...
%
% The \ctanpkg{libgreek} of |2022/11/11| extracted the |-TLF| postfix from the
% font family name, to reinsert it here explicitly, the options |serif/sans|
% deciding whether to use |LibertinusSerif-TLF| or |LibertinusSans-TLF| for
% reasons now escaping me.  I vaguely remember it was useful at some point
% during development.  Ah yes, now I remember this separation was for the
% handling of the |scale| option.  And we havn't one here.
%    \begin{macrocode}
\DeclareFontEncoding{LGR}{}{}
\DeclareSymbolFont{lgrmathup}{LGR}{\lgrmath@fontfamily}
                                   {\lgrmath@series}
                                   {\lgrmath@upshape}
\SetSymbolFont{lgrmathup}{bold}{LGR}{\lgrmath@fontfamily}
                                     {\lgrmath@boldseries}
                                     {\lgrmath@upshape}
\DeclareSymbolFont{lgrmathit}{LGR}{\lgrmath@fontfamily}
                                   {\lgrmath@series}
                                   {\lgrmath@itshape}
\SetSymbolFont{lgrmathit}{bold}{LGR}{\lgrmath@fontfamily}
                                     {\lgrmath@boldseries}
                                     {\lgrmath@itshape}
%    \end{macrocode}
% As all Greek letters are already available in \cs{...up} and \cs{...it}
% variants, it is indeed not immediately pressing to have math alphabets, so
% let's not do it by default.
%    \begin{macrocode}
\iflgrmath@alphabets
   \DeclareSymbolFontAlphabet{\lgrmathup}{lgrmathup}
   \DeclareSymbolFontAlphabet{\lgrmathit}{lgrmathit}
\fi
%    \end{macrocode}
% Definition of the `|up|' \cs{mathchar}'s.  There are 48 `standard'
% ones plus 8 extras and 11 with diacritics for a total of 67 ones.
%
% Hesitation whether I should declare with \cs{mathalpha} \emph{only} if
% |alphabets| is passed to the package.
%    \begin{macrocode}
\DeclareMathSymbol{\Alphaup}{\mathalpha}{lgrmathup}{65}
\DeclareMathSymbol{\Betaup}{\mathalpha}{lgrmathup}{66}
\DeclareMathSymbol{\Gammaup}{\mathalpha}{lgrmathup}{71}
\DeclareMathSymbol{\Deltaup}{\mathalpha}{lgrmathup}{68}
\DeclareMathSymbol{\Epsilonup}{\mathalpha}{lgrmathup}{69}
\DeclareMathSymbol{\Zetaup}{\mathalpha}{lgrmathup}{90}
\DeclareMathSymbol{\Etaup}{\mathalpha}{lgrmathup}{72}
\DeclareMathSymbol{\Thetaup}{\mathalpha}{lgrmathup}{74}
\DeclareMathSymbol{\Iotaup}{\mathalpha}{lgrmathup}{73}
\DeclareMathSymbol{\Kappaup}{\mathalpha}{lgrmathup}{75}
\DeclareMathSymbol{\Lambdaup}{\mathalpha}{lgrmathup}{76}
\DeclareMathSymbol{\Muup}{\mathalpha}{lgrmathup}{77}
\DeclareMathSymbol{\Nuup}{\mathalpha}{lgrmathup}{78}
\DeclareMathSymbol{\Xiup}{\mathalpha}{lgrmathup}{88}
\DeclareMathSymbol{\Omicronup}{\mathalpha}{lgrmathup}{79}
\DeclareMathSymbol{\Piup}{\mathalpha}{lgrmathup}{80}
\DeclareMathSymbol{\Rhoup}{\mathalpha}{lgrmathup}{82}
\DeclareMathSymbol{\Sigmaup}{\mathalpha}{lgrmathup}{83}
\DeclareMathSymbol{\Tauup}{\mathalpha}{lgrmathup}{84}
\DeclareMathSymbol{\Upsilonup}{\mathalpha}{lgrmathup}{85}
\DeclareMathSymbol{\Phiup}{\mathalpha}{lgrmathup}{70}
\DeclareMathSymbol{\Chiup}{\mathalpha}{lgrmathup}{81}
\DeclareMathSymbol{\Psiup}{\mathalpha}{lgrmathup}{89}
\DeclareMathSymbol{\Omegaup}{\mathalpha}{lgrmathup}{87}
\DeclareMathSymbol{\alphaup}{\mathalpha}{lgrmathup}{97}
\DeclareMathSymbol{\betaup}{\mathalpha}{lgrmathup}{98}
\DeclareMathSymbol{\gammaup}{\mathalpha}{lgrmathup}{103}
\DeclareMathSymbol{\deltaup}{\mathalpha}{lgrmathup}{100}
\DeclareMathSymbol{\epsilonup}{\mathalpha}{lgrmathup}{101}
\DeclareMathSymbol{\zetaup}{\mathalpha}{lgrmathup}{122}
\DeclareMathSymbol{\etaup}{\mathalpha}{lgrmathup}{104}
\DeclareMathSymbol{\thetaup}{\mathalpha}{lgrmathup}{106}
\DeclareMathSymbol{\iotaup}{\mathalpha}{lgrmathup}{105}
\DeclareMathSymbol{\kappaup}{\mathalpha}{lgrmathup}{107}
\DeclareMathSymbol{\lambdaup}{\mathalpha}{lgrmathup}{108}
\DeclareMathSymbol{\muup}{\mathalpha}{lgrmathup}{109}
\DeclareMathSymbol{\nuup}{\mathalpha}{lgrmathup}{110}
\DeclareMathSymbol{\xiup}{\mathalpha}{lgrmathup}{120}
\DeclareMathSymbol{\omicronup}{\mathalpha}{lgrmathup}{111}
\DeclareMathSymbol{\piup}{\mathalpha}{lgrmathup}{112}
\DeclareMathSymbol{\rhoup}{\mathalpha}{lgrmathup}{114}
\DeclareMathSymbol{\sigmaup}{\mathalpha}{lgrmathup}{115}
\DeclareMathSymbol{\tauup}{\mathalpha}{lgrmathup}{116}
\DeclareMathSymbol{\upsilonup}{\mathalpha}{lgrmathup}{117}
\DeclareMathSymbol{\phiup}{\mathalpha}{lgrmathup}{102}
\DeclareMathSymbol{\chiup}{\mathalpha}{lgrmathup}{113}
\DeclareMathSymbol{\psiup}{\mathalpha}{lgrmathup}{121}
\DeclareMathSymbol{\omegaup}{\mathalpha}{lgrmathup}{119}
%    \end{macrocode}
% Defintion of the `|it|' \cs{mathchar}'s.
%    \begin{macrocode}
\DeclareMathSymbol{\Alphait}{\mathalpha}{lgrmathit}{65}
\DeclareMathSymbol{\Betait}{\mathalpha}{lgrmathit}{66}
\DeclareMathSymbol{\Gammait}{\mathalpha}{lgrmathit}{71}
\DeclareMathSymbol{\Deltait}{\mathalpha}{lgrmathit}{68}
\DeclareMathSymbol{\Epsilonit}{\mathalpha}{lgrmathit}{69}
\DeclareMathSymbol{\Zetait}{\mathalpha}{lgrmathit}{90}
\DeclareMathSymbol{\Etait}{\mathalpha}{lgrmathit}{72}
\DeclareMathSymbol{\Thetait}{\mathalpha}{lgrmathit}{74}
\DeclareMathSymbol{\Iotait}{\mathalpha}{lgrmathit}{73}
\DeclareMathSymbol{\Kappait}{\mathalpha}{lgrmathit}{75}
\DeclareMathSymbol{\Lambdait}{\mathalpha}{lgrmathit}{76}
\DeclareMathSymbol{\Muit}{\mathalpha}{lgrmathit}{77}
\DeclareMathSymbol{\Nuit}{\mathalpha}{lgrmathit}{78}
\DeclareMathSymbol{\Xiit}{\mathalpha}{lgrmathit}{88}
\DeclareMathSymbol{\Omicronit}{\mathalpha}{lgrmathit}{79}
\DeclareMathSymbol{\Piit}{\mathalpha}{lgrmathit}{80}
\DeclareMathSymbol{\Rhoit}{\mathalpha}{lgrmathit}{82}
\DeclareMathSymbol{\Sigmait}{\mathalpha}{lgrmathit}{83}
\DeclareMathSymbol{\Tauit}{\mathalpha}{lgrmathit}{84}
\DeclareMathSymbol{\Upsilonit}{\mathalpha}{lgrmathit}{85}
\DeclareMathSymbol{\Phiit}{\mathalpha}{lgrmathit}{70}
\DeclareMathSymbol{\Chiit}{\mathalpha}{lgrmathit}{81}
\DeclareMathSymbol{\Psiit}{\mathalpha}{lgrmathit}{89}
\DeclareMathSymbol{\Omegait}{\mathalpha}{lgrmathit}{87}
\DeclareMathSymbol{\alphait}{\mathalpha}{lgrmathit}{97}
\DeclareMathSymbol{\betait}{\mathalpha}{lgrmathit}{98}
\DeclareMathSymbol{\gammait}{\mathalpha}{lgrmathit}{103}
\DeclareMathSymbol{\deltait}{\mathalpha}{lgrmathit}{100}
\DeclareMathSymbol{\epsilonit}{\mathalpha}{lgrmathit}{101}
\DeclareMathSymbol{\zetait}{\mathalpha}{lgrmathit}{122}
\DeclareMathSymbol{\etait}{\mathalpha}{lgrmathit}{104}
\DeclareMathSymbol{\thetait}{\mathalpha}{lgrmathit}{106}
\DeclareMathSymbol{\iotait}{\mathalpha}{lgrmathit}{105}
\DeclareMathSymbol{\kappait}{\mathalpha}{lgrmathit}{107}
\DeclareMathSymbol{\lambdait}{\mathalpha}{lgrmathit}{108}
\DeclareMathSymbol{\muit}{\mathalpha}{lgrmathit}{109}
\DeclareMathSymbol{\nuit}{\mathalpha}{lgrmathit}{110}
\DeclareMathSymbol{\xiit}{\mathalpha}{lgrmathit}{120}
\DeclareMathSymbol{\omicronit}{\mathalpha}{lgrmathit}{111}
\DeclareMathSymbol{\piit}{\mathalpha}{lgrmathit}{112}
\DeclareMathSymbol{\rhoit}{\mathalpha}{lgrmathit}{114}
\DeclareMathSymbol{\sigmait}{\mathalpha}{lgrmathit}{115}
\DeclareMathSymbol{\tauit}{\mathalpha}{lgrmathit}{116}
\DeclareMathSymbol{\upsilonit}{\mathalpha}{lgrmathit}{117}
\DeclareMathSymbol{\phiit}{\mathalpha}{lgrmathit}{102}
\DeclareMathSymbol{\chiit}{\mathalpha}{lgrmathit}{113}
\DeclareMathSymbol{\psiit}{\mathalpha}{lgrmathit}{121}
\DeclareMathSymbol{\omegait}{\mathalpha}{lgrmathit}{119}
%    \end{macrocode}
% Alternate shapes and other glyphs.
%    \begin{macrocode}
\DeclareMathSymbol{\varsigmaup}{\mathalpha}{lgrmathup}{99}
\DeclareMathSymbol{\varvarsigmaup}{\mathalpha}{lgrmathup}{6}
\DeclareMathSymbol{\varSigmaup}{\mathalpha}{lgrmathup}{22}
\DeclareMathSymbol{\Sampiup}{\mathalpha}{lgrmathup}{23}
\DeclareMathSymbol{\sampiup}{\mathalpha}{lgrmathup}{27}
\DeclareMathSymbol{\digammaup}{\mathalpha}{lgrmathup}{147}
\DeclareMathSymbol{\Digammaup}{\mathalpha}{lgrmathup}{195}
\DeclareMathSymbol{\koppaup}{\mathalpha}{lgrmathup}{18}
\DeclareMathSymbol{\varsigmait}{\mathalpha}{lgrmathit}{99}
\DeclareMathSymbol{\varvarsigmait}{\mathalpha}{lgrmathit}{6}
\DeclareMathSymbol{\varSigmait}{\mathalpha}{lgrmathit}{22}
\DeclareMathSymbol{\Sampiit}{\mathalpha}{lgrmathit}{23}
\DeclareMathSymbol{\sampiit}{\mathalpha}{lgrmathit}{27}
\DeclareMathSymbol{\digammait}{\mathalpha}{lgrmathit}{147}
\DeclareMathSymbol{\Digammait}{\mathalpha}{lgrmathit}{195}
\DeclareMathSymbol{\koppait}{\mathalpha}{lgrmathit}{18}
%    \end{macrocode}
% Some glyphs with diacritics.  I decided not to keep this
% in package.  Let's wait for extremely improbable feature request, as I won't
% do the feature request and will probably remain the sole user.  Actually I
% don't think I will ever use this package as I context I may think of where
% it could be useful are those where I would use \ctanpkg{mathastext} and its
% |LGRgreek| option and \cs{MTgreekfont} command...
%    \begin{macrocode}
% \DeclareMathSymbol{\alphatonosup}{\mathalpha}{lgrmathup}{136}
% \DeclareMathSymbol{\epsilontonosup}{\mathalpha}{lgrmathup}{232}
% \DeclareMathSymbol{\etatonosup}{\mathalpha}{lgrmathup}{160}
% \DeclareMathSymbol{\iotatonosup}{\mathalpha}{lgrmathup}{208}
% \DeclareMathSymbol{\omicrontonosup}{\mathalpha}{lgrmathup}{236}
% \DeclareMathSymbol{\upsilontonosup}{\mathalpha}{lgrmathup}{212}
% \DeclareMathSymbol{\omegatonosup}{\mathalpha}{lgrmathup}{184}
% \DeclareMathSymbol{\upsilondieresistonosup}{\mathalpha}{lgrmathup}{246}
% \DeclareMathSymbol{\iotadieresisup}{\mathalpha}{lgrmathup}{240}
% \DeclareMathSymbol{\iotadieresistonosup}{\mathalpha}{lgrmathup}{242}
% \DeclareMathSymbol{\upsilondieresisup}{\mathalpha}{lgrmathup}{244}
% \DeclareMathSymbol{\alphatonosit}{\mathalpha}{lgrmathit}{136}
% \DeclareMathSymbol{\epsilontonosit}{\mathalpha}{lgrmathit}{232}
% \DeclareMathSymbol{\etatonosit}{\mathalpha}{lgrmathit}{160}
% \DeclareMathSymbol{\iotatonosit}{\mathalpha}{lgrmathit}{208}
% \DeclareMathSymbol{\omicrontonosit}{\mathalpha}{lgrmathit}{236}
% \DeclareMathSymbol{\upsilontonosit}{\mathalpha}{lgrmathit}{212}
% \DeclareMathSymbol{\omegatonosit}{\mathalpha}{lgrmathit}{184}
% \DeclareMathSymbol{\upsilondieresistonosit}{\mathalpha}{lgrmathit}{246}
% \DeclareMathSymbol{\iotadieresisit}{\mathalpha}{lgrmathit}{240}
% \DeclareMathSymbol{\iotadieresistonosit}{\mathalpha}{lgrmathit}{242}
% \DeclareMathSymbol{\upsilondieresisit}{\mathalpha}{lgrmathit}{244}
%    \end{macrocode}
% Definition of the \cs{mathchar}'s without `|up/it|' postfix. There are 27
% uppercase and 40 lowercase letters, for a total of 67 glyphs.  Actually,
% I had done some work with LGR in September 2011.  I kept
% the file around.  But
% at no point did I go back to check if I had done exhaustive work in 2011 and
% whether some other glyphs could be accounted for by LGR (not using
% ligatures) (I did re-check an
% old file about the LGR encoding I had from that 2011 work, but did not try
% to check for updates).  Anyway, it is very doubtful whether it made any
% sense for \lgrmath to define control sequences for Greek letters with
% diacritics...
%    \begin{macrocode}
\def\lgrmath@setgreekcs{%
    \iflgrmath@upper@up
       \let\Alpha\Alphaup
       \let\Beta\Betaup
       \let\Gamma\Gammaup
       \let\Delta\Deltaup
       \let\Epsilon\Epsilonup
       \let\Zeta\Zetaup
       \let\Eta\Etaup
       \let\Theta\Thetaup
       \let\Iota\Iotaup
       \let\Kappa\Kappaup
       \let\Lambda\Lambdaup
       \let\Mu\Muup
       \let\Nu\Nuup
       \let\Xi\Xiup
       \let\Omicron\Omicronup
       \let\Pi\Piup
       \let\Rho\Rhoup
       \let\Sigma\Sigmaup
       \let\Tau\Tauup
       \let\Upsilon\Upsilonup
       \let\Phi\Phiup
       \let\Chi\Chiup
       \let\Psi\Psiup
       \let\Omega\Omegaup
       \let\Sampi\Sampiup
       \let\Digamma\Digammaup
       \let\varSigma\varSigmaup
    \else
       \let\Alpha\Alphait
       \let\Beta\Betait
       \let\Gamma\Gammait
       \let\Delta\Deltait
       \let\Epsilon\Epsilonit
       \let\Zeta\Zetait
       \let\Eta\Etait
       \let\Theta\Thetait
       \let\Iota\Iotait
       \let\Kappa\Kappait
       \let\Lambda\Lambdait
       \let\Mu\Muit
       \let\Nu\Nuit
       \let\Xi\Xiit
       \let\Omicron\Omicronit
       \let\Pi\Piit
       \let\Rho\Rhoit
       \let\Sigma\Sigmait
       \let\Tau\Tauit
       \let\Upsilon\Upsilonit
       \let\Phi\Phiit
       \let\Chi\Chiit
       \let\Psi\Psiit
       \let\Omega\Omegait
       \let\Sampi\Sampiit
       \let\Digamma\Digammait
       \let\varSigma\varSigmait
    \fi
    \iflgrmath@lower@up
       \let\alpha\alphaup
       \let\beta\betaup
       \let\gamma\gammaup
       \let\delta\deltaup
       \let\epsilon\epsilonup
       \let\zeta\zetaup
       \let\eta\etaup
       \let\theta\thetaup
       \let\iota\iotaup
       \let\kappa\kappaup
       \let\lambda\lambdaup
       \let\mu\muup
       \let\nu\nuup
       \let\xi\xiup
       \let\omicron\omicronup
       \let\pi\piup
       \let\rho\rhoup
       \let\sigma\sigmaup
       \let\tau\tauup
       \let\upsilon\upsilonup
       \let\phi\phiup
       \let\chi\chiup
       \let\psi\psiup
       \let\omega\omegaup
       \let\varsigma\varsigmaup
       \let\varvarsigma\varvarsigmaup
       \let\sampi\sampiup
       \let\digamma\digammaup
       \let\koppa\koppaup
% \let\alphatonos\alphatonosup
% \let\epsilontonos\epsilontonosup
% \let\etatonos\etatonosup
% \let\iotatonos\iotatonosup
% \let\omicrontonos\omicrontonosup
% \let\upsilontonos\upsilontonosup
% \let\omegatonos\omegatonosup
% \let\upsilondieresistonos\upsilondieresistonosup
% \let\iotadieresis\iotadieresisup
% \let\iotadieresistonos\iotadieresistonosup
% \let\upsilondieresis\upsilondieresisup
    \else
       \let\alpha\alphait
       \let\beta\betait
       \let\gamma\gammait
       \let\delta\deltait
       \let\epsilon\epsilonit
       \let\zeta\zetait
       \let\eta\etait
       \let\theta\thetait
       \let\iota\iotait
       \let\kappa\kappait
       \let\lambda\lambdait
       \let\mu\muit
       \let\nu\nuit
       \let\xi\xiit
       \let\omicron\omicronit
       \let\pi\piit
       \let\rho\rhoit
       \let\sigma\sigmait
       \let\tau\tauit
       \let\upsilon\upsilonit
       \let\phi\phiit
       \let\chi\chiit
       \let\psi\psiit
       \let\omega\omegait
       \let\varsigma\varsigmait
       \let\varvarsigma\varvarsigmait
       \let\sampi\sampiit
       \let\digamma\digammait
       \let\koppa\koppait
% \let\alphatonos\alphatonosit
% \let\epsilontonos\epsilontonosit
% \let\etatonos\etatonosit
% \let\iotatonos\iotatonosit
% \let\omicrontonos\omicrontonosit
% \let\upsilontonos\upsilontonosit
% \let\omegatonos\omegatonosit
% \let\upsilondieresistonos\upsilondieresistonosit
% \let\iotadieresis\iotadieresisit
% \let\iotadieresistonos\iotadieresistonosit
% \let\upsilondieresis\upsilondieresisit
    \fi
}%
\lgrmath@setgreekcs
%    \end{macrocode}
% Finally we define \cs{lgrmathgreektable} and \cs{lgrmathgreektableextra}.
% It would arguably be neater to code this using Latin letters in place of the
% \cs{char}'s but I had this ready from |libgreek.dtx| and it is important in
% life to optimize time allocations.
%    \begin{macrocode}
\def\lgrmathgreektable#1#2#3{%
\begingroup
% ah que TeX est mieux avec tt
\def\s##1{{\usefont{T1}{mlmtt}{m}{n}\string##1}}%
\usefont{LGR}{#1}{#2}{#3}%
\begin{tabular}{|lc|lc|lc|lc|}
\hline
\s\Alpha&\char65   &\s\Nu&\char78      &\s\alpha&\char97    &\s\nu&\char110 \\
\s\Beta&\char66    &\s\Xi&\char88      &\s\beta&\char98     &\s\xi&\char120 \\
\s\Gamma&\char71   &\s\Omicron&\char79 &\s\gamma&\char103   &\s\omicron&\char111 \\
\s\Delta&\char68   &\s\Pi&\char80      &\s\delta&\char100   &\s\pi&\char112 \\
\s\Epsilon&\char69 &\s\Rho&\char82     &\s\epsilon&\char101 &\s\rho&\char114 \\
\s\Zeta&\char90    &\s\Sigma&\char83   &\s\zeta&\char122    &\s\sigma&\char115 \\
\s\Eta&\char72     &\s\Tau&\char84     &\s\eta&\char104     &\s\tau&\char116 \\
\s\Theta&\char74   &\s\Upsilon&\char85 &\s\theta&\char106   &\s\upsilon&\char117 \\
\s\Iota&\char73    &\s\Phi&\char70     &\s\iota&\char105    &\s\phi&\char102 \\
\s\Kappa&\char75   &\s\Chi&\char81     &\s\kappa&\char107   &\s\chi&\char113 \\
\s\Lambda&\char76  &\s\Psi&\char89     &\s\lambda&\char108  &\s\psi&\char121 \\
\s\Mu&\char77      &\s\Omega&\char87   &\s\mu&\char109      &\s\omega&\char119 \\
\hline
\end{tabular}
\endgroup
}%
\def\lgrmathgreektableextra#1#2#3{%
\begingroup
% ah que TeX est mieux avec tt
\def\s##1{{\usefont{T1}{mlmtt}{m}{n}\string##1}}%
\usefont{LGR}{#1}{#2}{#3}%
\begin{tabular}{lclclc}
\s\varsigma&\char99   &\s\digamma&\char147 &\s\varSigma&\char22 \\
\s\varvarsigma&\char6 &\s\koppa&\char18    &\s\Sampi&\char23    \\
\s\sampi&\char27      &        &           &\s\Digamma&\char195   
\end{tabular}
\endgroup
}%
%    \end{macrocode}
% And we have now reached the end of the \lgrmath package code.  The actual
% |.sty| file will contain an \cs{endinput} added by the DocStrip extraction.
%\MakePercentComment
%</sty>
%<*dtx>
\Finale
%%
%% End of file `lgrmath.dtx'.
